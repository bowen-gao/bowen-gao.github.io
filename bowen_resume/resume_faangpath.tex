\documentclass{resume} % Use the custom resume.cls style

\usepackage[left=0.5in,top=0.5in,right=0.5in,bottom=0.5in]{geometry} % Document margins
\usepackage{bibentry}
\usepackage{natbib}
\usepackage{enumitem} % Better list control
\usepackage{xcolor} % For colors

% Define color scheme
\definecolor{darkblue}{RGB}{0,51,102}
\definecolor{mediumgray}{RGB}{80,80,80}

\newcommand{\tab}[1]{\hspace{.2667\textwidth}\rlap{#1}} 
\newcommand{\itab}[1]{\hspace{0em}\rlap{#1}}
\newcommand{\highlightAuthor}[1]{\textbf{#1}}
\newcommand{\ec}{\textsuperscript{*}}

% Update hyperref colors
\hypersetup{
    colorlinks=true,
    linkcolor=darkblue,
    filecolor=darkblue,      
    urlcolor=darkblue,
}

\name{Bowen Gao} % Your name
% You can merge both of these into a single line, if you do not have a website.
\address{Beijing, China \\ +86 13810620833} 
\address{\href{mailto:billgao0111@gmail.com}{billgao0111@gmail.com} \\ \href{https://www.linkedin.com/in/bgao}{LinkedIn} \\ \href{https://scholar.google.com/citations?user=cTGzVe8AAAAJ&hl=en}{Google Scholar}}

\setlength{\bibsep}{0.0pt plus 0.3ex}
\setlength{\bibhang}{0pt}

\begin{document}

%----------------------------------------------------------------------------------------
%	EDUCATION SECTION
%----------------------------------------------------------------------------------------

\nobibliography{cv}
\bibliographystyle{unsrt}

\begin{rSection}{Education}

{\bf Tsinghua University} \hfill {\bf Beijing, China} 
\\
Ph.D. in Computer Science and Technology \hfill {\em August 2024 - Present} 
 \begin{itemize}[leftmargin=1.5em, itemsep=2pt, parsep=0pt, topsep=3pt]
     \item Supervised by Professor \textbf{Ya-Qin Zhang} and Professor \textbf{Yanyan Lan}
 \end{itemize}

{\bf California Institute of Technology} \hfill {\bf Pasadena, U.S.}
\\
Master of Electrical Engineering \hfill {\em October 2019 - June 2021}
 \begin{itemize}[leftmargin=1.5em, itemsep=2pt, parsep=0pt, topsep=3pt]
     \item GPA: 4.2 / 4.3
     \item Advised by Professor Yaser Abu-Mostafa and Professor Yisong Yue
 \end{itemize}

{\bf University of Toronto} \hfill {\bf Toronto, Canada}
\\
Bachelor of Computer Science \hfill {\em September 2014 - June 2019}
 \begin{itemize}[leftmargin=1.5em, itemsep=2pt, parsep=0pt, topsep=3pt]
     \item GPA: 3.85 / 4.0
     \item Dean's Honour List for all academic years
     \item Graduated with Highest Honors
 \end{itemize}

\end{rSection}


\begin{rSection}{Research Interest}

My research focuses on leveraging artificial intelligence for drug discovery (AIDD), with a particular emphasis on developing and applying deep learning models for the representation and generation of small molecules and proteins. I aim to build \textbf{data-centric} methods to address the data scarcity problem in the AIDD domain.
    
\end{rSection}

\vspace{-0.5em} % Reduce spacing before next section

%----------------------------------------------------------------------------------------
%	WORK EXPERIENCE SECTION
%----------------------------------------------------------------------------------------

\begin{rSection}{Employments}

\textbf{Institute for AI Industry Research, Tsinghua University (AIR)} \hfill {\em September 2022 - August 2024}
\\
Full Time Research Engineer

\vspace{0.5em}

\textbf{Applied Machine Learning (AML) at ByteDance} \hfill {\em July 2021 - September 2022}
\\
Full Time Machine Learning Engineer

\vspace{0.5em}

\textbf{Uber ATG} \hfill {\em June 2020 - September 2020}
\\
Autonomous Driving Algorithm Intern

\end{rSection}

\begin{rSection}{Publications}

\begin{enumerate}[leftmargin=2em, itemsep=3pt, parsep=0pt, topsep=3pt]
    \item \bibentry{jia2024deep}.
    \item \bibentry{gao2025cidd}.
    \item \bibentry{zhu2025aanet}.
    \item \bibentry{huang2024siu}.
    \item \bibentry{gao2025reframing}.
    \item \bibentry{gaorethinking}.
    \item \bibentry{gaoself}.
    \item \bibentry{gao2024drugclip}.
    \item \bibentry{qiang2023coarse}. 
\end{enumerate}

\end{rSection}

\begin{rSection}{Preprints}

\begin{enumerate}[leftmargin=2em, itemsep=3pt, parsep=0pt, topsep=3pt]
    \item \bibentry{zhu2025coder}
    \item \bibentry{gao2025pharmagents}
    \item \bibentry{mo2024multi}.
    
    %\item \bibentry{Gao2020.11.17.387860}.
\end{enumerate}

\end{rSection}



%----------------------------------------------------------------------------------------
%	INTERN EXPERIENCE SECTION
%----------------------------------------------------------------------------------------

% \begin{rSection}{Internship Experience}

% % \textbf{Research Intern} \hfill June 2020 - Nov 2020 \\
% % Wang Lab@University of Washington
% %  \begin{itemize}
% %     \itemsep -3pt {} 
% %      \item Collaborated with Dr. Sheng Wang on using adversarial learning for tumor stratification.
% %      \item Employed a Generative Adversarial Network (GAN) to train a model capable of aligning different cancer types, followed by clustering in the aligned space instead of the original space; utilized consensus clustering for robust results.
% %      \item Outperformed baseline in tumor stratification for 8 out of 10 cancer types with significant log-rank test p-values (p-value $<$ 0.01); achieved an AUROC of 0.74 for cancer gene identification, compared to 0.53 for the baseline.
% %      \item Served as a major contributor in code development, experimentation, and paper writing; submitted as first author to RECOMB 2021 with the title: "cancerAlign: Stratifying tumors by unsupervised alignment across cancer types".
% %  \end{itemize}

% \textbf{Machine Learning Engineer Intern} \hfill June 2020 - Sep 2020\\
% Uber ATG
%  % \begin{itemize}
%  %    \itemsep -3pt {} 
%  %     \item Worked with the Perception team at Uber Advanced Technology Group.
%  %     \item Integrated a new birds-eye-view (BEV) detection head into the existing joint detection and segmentation model.
%  %     \item Developed a feature transformer to resample image feature maps for creating BEV features, allowing 3D detection loss to operate directly on the BEV feature map.
%  %     \item Employed focal loss for class loss and Huber loss for box offset loss, while incorporating hourglass layers to enhance the model.
%  %     \item Achieved over 94\% average precision for short-range (0-25m) pedestrian detection in 3D.
%  % \end{itemize}

% \textbf{C/C++ Compiler Developer Intern} \hfill May 2017 - June 2018\\
% IBM Canada
%  % \begin{itemize}
%  %    \itemsep -3pt {} 
%  %     \item Implemented new features for C++ compiler in accordance with C++ Standards.
%  %     \item Addressed and resolved various compiler front-end defects, focusing on command-line compiler options.
%  %     \item Refactored code to manage compiler behavior across different operating systems, including AIX and zOS.
%  % \end{itemize}

% \end{rSection}

%----------------------------------------------------------------------------------------
%	PUBLICATIONS SECTION
%----------------------------------------------------------------------------------------


\begin{rSection}{Academic Services}

 \begin{itemize}[leftmargin=1.5em, itemsep=2pt, parsep=0pt, topsep=3pt]
    \item Reviewer for International Conference on Learning Representation (ICLR) 2025, 2026
    \item Reviewer for Annual Conference on Artificial Intelligence (AAAI) 2026
    \item Reviewer for Neural Information Processing Systems (NeurIPS) 2024, 2025
    \item Reviewer for International Conference on Machine Learning (ICML) 2025
    \item Reviewer for International Conference on Artificial Intelligence and Statistics (AISTATS) 2025
    \item Reviewer for IEEE Transactions on Neural Networks and Learning Systems (TNNLS)
 \end{itemize}

\end{rSection}


\end{document}
